\documentclass{article}
\usepackage{booktabs}
\usepackage[UTF8]{ctex}
\usepackage{enumitem}
\usepackage{graphicx}
\usepackage{float}
\usepackage{subfigure}
\usepackage{amsmath,amsfonts}
\usepackage{hyperref}

\begin{document}
\sloppy % 解决中英文混排的断行问题

\title{《图书馆管理系统》实验报告}
\author{郑泓东 21311570}
\date{\today}

\maketitle

\section{登录界面}

\subsection{输入错误用户名或者密码,能够提示相关错误}

% 图片
\begin{figure}[H]
    \centering
    \includegraphics[width=1\textwidth]{../pic/login_fail.png}
    \caption{Login Fail}
    \label{fig:login_fail}
\end{figure}

\subsection{输入正确用户名和密码,能够登录}

\section{系统主界面}
提供各个功能模块的访问链接/按钮

\begin{figure}[H]
    \centering
    \includegraphics[width=1\textwidth]{../pic/admin_main.png}
    \caption{Administator Main}
    \label{fig:admin_main}
\end{figure}


\section{读者信息管理}

\section{图书借阅管理}
能够为读者办理借阅图书,并且使用事务实现业务;实现借阅图书业务后,在数据库留下借阅记录、更新图书馆藏信息、更新图书基本信息;更新读者信息表中的已借数量。

% 参考文献
\begin{thebibliography}{99}
    \bibitem{ref1} GUSLOVESMATH. \href{https://www.kaggle.com/code/guslovesmath/los-angeles-crime-data-quick-eda}{Los Angeles Crime Data Quick EDA}.
    \bibitem{ref2} ANDREI SAFRONOV. \href{https://www.kaggle.com/code/safronov00/crimesolver-predictor#2.-Clean-Data}{CrimeSolver Predictor}.
\end{thebibliography}


\end{document}